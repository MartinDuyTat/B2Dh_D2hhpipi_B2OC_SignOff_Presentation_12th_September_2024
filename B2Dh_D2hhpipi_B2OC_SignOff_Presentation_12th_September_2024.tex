%%%%%%%%%%%%%%%%%%%%%%%%%%%%%%%%%%%%%%%%%%%%%%%%%%%%%%%%%%%%%%%%%%%%%%
% Overleaf (WriteLaTeX) Example: Molecular Chemistry Presentation
%
% Source: http://www.overleaf.com
%
% In these slides we show how Overleaf can be used with standard 
% chemistry packages to easily create professional presentations.
% 
% Feel free to distribute this example, but please keep the referral
% to overleaf.com
% 
%%%%%%%%%%%%%%%%%%%%%%%%%%%%%%%%%%%%%%%%%%%%%%%%%%%%%%%%%%%%%%%%%%%%%%

\documentclass[xcolor={dvipsnames}]{beamer}

\mode<presentation>
{
  \usetheme{Madrid}       % or try default, Darmstadt, Warsaw, ...
  \usecolortheme{default} % or try albatross, beaver, crane, ...
  \usefonttheme{default}    % or try default, structurebold, ...
  \setbeamertemplate{navigation symbols}{}
  \setbeamertemplate{caption}[numbered]
} 

\usepackage[english]{babel}
\usepackage[utf8x]{inputenc}
\usepackage{graphicx}
\usepackage{hyperref}
  \hypersetup{colorlinks=true}
  \hypersetup{urlcolor=blue}
  \hypersetup{linkcolor = .}
\usepackage{xcolor}
\usepackage{siunitx}
  \sisetup{separate-uncertainty = true}
\DeclareSIUnit\barn{b}
\usepackage{physics}
\usepackage[font=small,labelfont=bf]{caption}
\usepackage{subcaption}
\usepackage[en-GB]{datetime2}
\usepackage{overpic}
\usepackage{feynmp}
\DeclareGraphicsRule{*}{mps}{*}{}
\usepackage{scalerel}
\newcommand{\mylbrace}[2]{\vspace{#2pt}\hspace{6pt}\scaleleftright[\dimexpr5pt+#1\dimexpr0.06pt]{\lbrace}{\rule[\dimexpr2pt-#1\dimexpr0.5pt]{-4pt}{#1pt}}{.}}
\newcommand{\myrbrace}[2]{\vspace{#2pt}\scaleleftright[\dimexpr5pt+#1\dimexpr0.06pt]{.}{\rule[\dimexpr2pt-#1\dimexpr0.5pt]{-4pt}{#1pt}}{\rbrace}\hspace{6pt}}

% Trim in percent
\usepackage{adjustbox}

% No "Figure" prefix
\setbeamertemplate{caption}{\raggedright\insertcaption\par}

% Nice decay amplitude diagrams
\usepackage{amsmath,amssymb,tikz-cd}

% Strike out text
\usepackage[normalem]{ulem}

% For figures with text overlay
\usepackage{overpic}

% Arrows
\usepackage{tikz}
\newcommand{\tikzmark}[1]{\tikz[remember picture] \node[coordinate] (#1) {#1};}

% Colourbox with line breaks
\newcommand{\cbox}[2][lime!20]{%
  \colorbox{#1}{\parbox{\dimexpr\linewidth-2\fboxsep}{\strut #2\strut}}%
}

% Vector arrows
\usepackage[pdftex]{pict2e}

% Checkmark symbol
\def\checkmark{\tikz\fill[scale=0.4](0,.35) -- (.25,0) -- (1,.7) -- (.25,.15) -- cycle;} 

% Here's where the presentation starts, with the info for the title slide
\title[B2OC]{Model-independent measurement of the CKM angle \texorpdfstring{$\gamma$}{gamma} in \texorpdfstring{$B^\pm\to[h^+h^-\pi^+\pi^-]_Dh^\pm$}{B2DhD2hhpipi} decays}

\author[Martin Tat]{Sneha Malde \inst{1}, Claire Prouve \inst{2}, Jonas Rademacker \inst{3}, \textbf{Martin Tat}\inst{1,5}, Ben Westhenry \inst{3}, Mark Whitehead \inst{4}, Guy Wilkinson \inst{1}}
\institute[University of Oxford]{\tiny \inst{1} University of Oxford, \inst{2} Universidade da Coruña, \inst{3} University of Bristol, \inst{4} University of Glasgow, \inst{5} Universit\"at Heidelberg}
\date{12th September 2024}

\titlegraphic{\includegraphics[height = 1.5cm]{lhcb.jpg}\hspace{0.0cm}~%
              \includegraphics[height = 1.5cm]{HeidelbergLogo.pdf}\hspace{0.0cm}~%
              \includegraphics[height = 1.5cm]{OxfordLogo.pdf}\hspace{0.0cm}~%
              \includegraphics[height = 1.5cm]{Bristol_logo.jpg}}

\begin{document}

\begin{frame}
  \titlepage
\end{frame}

% These three lines create an automatically generated table of contents.
\begin{frame}{Outline}
  \tableofcontents
\end{frame}

\begin{frame}{Acknowledgements}
  \begin{center}
    Thanks to:
  \end{center}
  \begin{itemize}
    \item{Anton Poluektov}
    \item{Wenbin Qian}
    \item{Resmi Puthumanaillam}
  \end{itemize}
  \vspace{0.3cm}
  \begin{center}
    for their helpful comments, suggestions and feedback during WG review
  \end{center}
\end{frame}

\section{Introduction to \texorpdfstring{$\gamma$}{gamma} and \texorpdfstring{$C\!P$}{CP} violation}
\begin{frame}{Introduction to $\gamma$ and $C\!P$ violation}
  \begin{itemize}
    \setlength\itemsep{0.3em}
    \item{CPV in SM is described by the Unitary Triangle, with angles $\alpha$, $\beta$, $\gamma$}
    \item{The angle $\gamma = \text{arg}\Big(-\frac{V^{\phantom{*}}_{ud}V^*_{ub}}{V^{\phantom{*}}_{cd}V^*_{cb}}\Big)$ is very important:}
    \begin{enumerate}
    \setlength\itemsep{0.2em}
      \item{Negligible theoretical uncertainties: Ideal SM benchmark}
      \item{Accessible at tree level: Indirectly probe New Physics that enter loops}
      \item{Compare with a global CKM fit: Is the Unitary Triangle a triangle?}
    \end{enumerate}
  \end{itemize}
  \vspace{-0.2cm}
  \begin{figure}
    \centering
    \begin{subfigure}{0.5\textwidth}
      \centering
      \includegraphics[width = 1.0\textwidth]{Plots/ckmfitter_tree.png}
      \caption{Tree level: $\gamma = \big(72.1^{+5.4}_{-5.7}\big)^\circ$}
    \end{subfigure}%
    \begin{subfigure}{0.5\textwidth}
      \centering
      \includegraphics[width = 1.0\textwidth]{Plots/ckmfitter_loop.png}
      \caption{Loop level: $\gamma = \big(65.5^{+1.1}_{-2.7}\big)^\circ$}
    \end{subfigure}
    \vspace{-0.8cm}
    \captionsetup{justification=centering}
    \caption*{\centering\tiny CKMfitter Group (J. Charles et al.), Eur. Phys. J. C41, 1-131 (2005), updated results and plots available at: \href{http://ckmfitter.in2p3.fr}{http://ckmfitter.in2p3.fr}}
  \end{figure}
\end{frame}

\begin{frame}{Sensitivity through interference}
  \begin{center}
    \Large Measure $\gamma$ through interference effects in $B^\pm\to DK^\pm$
  \end{center}
  \begin{figure}[H]
    \centering
    \begin{subfigure}{0.5\textwidth}
      \centering
      \begin{fmffile}{fgraph/fgraph_BtoDK1}
        \setlength{\unitlength}{0.4cm}
        \begin{fmfgraph*}(6,6)
          \fmfstraight
          \fmfleft{i1,B,i2,t1,t2,t3,t9,t10}
          \fmfright{o1,D,o2,t4,t5,o3,K,o4}
          \fmflabel{$\bar{u}$}{i1}
          \fmflabel{$b$}{i2}
          \fmfv{l.d=20,l.a=180,l={$B^-$\mylbrace{30}{-8}}}{B}
          \fmflabel{$\bar{u}$}{o1}
          \fmflabel{$c$}{o2}
          \fmflabel{$\bar{u}$}{o3}
          \fmflabel{$s$}{o4}
          \fmfv{l.d=15,l.a=0,l={\myrbrace{30}{-12}}$D^0$}{D}
          \fmfv{l.d=15,l.a=0,l={\myrbrace{30}{11}}$K^-$}{K}
          \fmf{fermion}{o1,i1}
          \fmf{fermion,tension=1.5}{i2,v1}
          \fmf{fermion}{v1,o2}
          \fmf{phantom,tension=1.5}{t9,v2}
          \fmf{boson,label=$W$,label.side=left,tension=0}{v1,v2}
          \fmf{fermion}{v2,o4}
          \fmf{fermion}{o3,v2}
        \end{fmfgraph*}
      \end{fmffile}
      \vspace{0.5cm}
      \caption*{Favoured $B^-\to D^0K^-$}
    \end{subfigure}%
    \begin{subfigure}{0.5\textwidth}
      \centering
      \begin{fmffile}{fgraph/fgraph_BtoDK2}
        \setlength{\unitlength}{0.4cm}
        \begin{fmfgraph*}(6,6)
          \fmfstraight
          \fmfleft{i1,t1,t2,B,t9,t10,i2}
          \fmfright{o1,K,o2,t4,t5,o3,D,o4}
          \fmflabel{$\bar{u}$}{i1}
          \fmflabel{$b$}{i2}
          \fmfv{l.d=20,l.a=180,l={$B^-$\mylbrace{100}{-8}}}{B}
          \fmflabel{$\bar{u}$}{o1}
          \fmflabel{$s$}{o2}
          \fmflabel{$\bar{c}$}{o3}
          \fmflabel{$u$}{o4}
          \fmfv{l.d=15,l.a=0,l={\myrbrace{30}{13}}$\bar{D^0}$}{D}
          \fmfv{l.d=15,l.a=0,l={\myrbrace{30}{-13}}$K^-$}{K}
          \fmf{fermion}{o1,i1}
          \fmf{fermion,tension=1.5}{i2,v1}
          \fmf{fermion}{v1,o4}
          \fmf{phantom,tension=1.5}{t2,v2}
          \fmf{boson,label=$W$,label.side=left,tension=0}{v1,v2}
          \fmf{fermion}{v2,o2}
          \fmf{fermion}{o3,v2}
        \end{fmfgraph*}
      \end{fmffile}
      \vspace{0.5cm}
      \caption*{Suppressed $B^-\to\bar{D^0}K^-$}
    \end{subfigure}
  \end{figure}
  \vspace{-0.3cm}
  \begin{itemize}
    \item{Superposition of $D^0$ and $\bar{D^0}$}
    \begin{itemize}
      \item{Consider $D^0$/$\bar{D^0}$ decays to the same final state, such as $D\to K^+K^-$}
    \end{itemize}
    \item{$b\to u\bar{c}s$ and $b\to c\bar{u}s$ interference $\to$ Sensitivity to $\gamma$}
  \end{itemize}
  \vspace{-0.3cm}
  \begin{center}
    $\mathcal{A}(B^-)=\mathcal{A}_B\Big(\mathcal{A}_{D^0} + r_Be^{i(\delta_B - \gamma)}\mathcal{A}_{\bar{D^0}}\Big)$ \\
    $\mathcal{A}(B^+)=\mathcal{A}_B\Big(\mathcal{A}_{\bar{D^0}} + r_Be^{i(\delta_B + \gamma)}\mathcal{A}_{D^0}\Big)$ \\
  \end{center}
\end{frame}

\begin{frame}{The BPGGSZ method}
\begin{center}
    \begin{minipage}{0.6\textwidth}
      \begin{block}{Event yield in bin $i$}
        \footnotesize
        $N^-_i = h_{B^-}\big(F_i + (x_-^2 + y_-^2)\bar{F_i} + 2\sqrt{F_i\bar{F_i}}(x_-c_i + y_-s_i)\big)$ \\
        $N^+_{-i} = h_{B^+}\big(F_i + (x_+^2 + y_+^2)\bar{F_i} + 2\sqrt{F_i\bar{F_i}}(x_+c_i + y_+s_i)\big)$
      \end{block}
    \end{minipage}
  \end{center}
  \begin{itemize}
    \item{CP observables:}
    \begin{itemize}
      \item{$x_\pm^{DK} = r_B^{DK}\cos(\delta_B^{DK}\pm\gamma)$, \quad $y_\pm^{DK} = r_B^{DK}\sin(\delta_B^{DK}\pm\gamma)$}
      \item{$x_\xi^{D\pi} = \Re(\xi^{D\pi})$, $y_\xi^{D\pi} = \Im(\xi^{D\pi})$ $\quad\quad\Big(\xi^{D\pi} = \frac{r_B^{D\pi}}{r_B^{DK}}e^{i(\delta_B^{D\pi} - \delta_B^{DK})}\Big)$}
    \end{itemize}
    \item{Fractional bin yield:}
    \begin{itemize}
      \item{$F_i = \frac{\int_i\dd{\Phi}|\mathcal{A}(D^0)|^2}{\sum_j\int_j\dd{\Phi}\abs{\mathcal{A}(D^0)}^2}$}
      \item{Floated in the fit, mostly constrained by $B^\pm\to D\pi^\pm$}
    \end{itemize}
  \end{itemize}
  \begin{itemize}
    \item{Amplitude-averaged strong phases:}
    \begin{center}
      $c_i = \frac{\int_i\dd{\Phi}|\mathcal{A}(D^0)||\mathcal{A}(\bar{D^0})|\cos(\delta_D)}{\sqrt{\int_i\dd{\Phi}\abs{\mathcal{A}(D^0)}^2\int_i\dd{\Phi}\abs{\mathcal{A}(\bar{D^0})}^2}}$ \quad $s_i = \frac{\int_i\dd{\Phi}|\mathcal{A}(D^0)||\mathcal{A}(\bar{D^0})|\sin(\delta_D)}{\sqrt{\int_i\dd{\Phi}\abs{\mathcal{A}(D^0)}^2\int_i\dd{\Phi}\abs{\mathcal{A}(\bar{D^0})}^2}}$
    \end{center}
  \end{itemize}
\end{frame}

\begin{frame}{$D^0\to K^+K^-\pi^+\pi^-$ binning scheme}
  \begin{itemize}
    \setlength\itemsep{0.5em}
    \item{Interpretation of $\gamma$ from the multi-body charm decays require external inputs of the charm strong-phase differences}
    \item{Measure model-independent strong-phases at a charm factory, such as BESIII, using an optimised binning scheme}
  \end{itemize}
  \begin{figure}
    \centering
    \begin{subfigure}{0.5\textwidth}
      \centering
      \includegraphics[height = 3.5cm]{Plots/BinningSchemePlot_4Bins.pdf}
      \vspace{-0.3cm}
      \caption*{4 bins}
    \end{subfigure}%
    \begin{subfigure}{0.5\textwidth}
      \centering
      \includegraphics[height = 3.5cm]{Plots/BinningSchemePlot_8Bins.pdf}
      \vspace{-0.3cm}
      \caption*{8 bins}
    \end{subfigure}
    \caption*{$D^0\to K^+K^-\pi^+\pi^-$ binning scheme}
  \end{figure}
\end{frame}

\begin{frame}{External strong-phase inputs}
  \begin{center}
    During B2OC WG review, preliminary results from BESIII have been used
  \end{center}
  \begin{itemize}
    \item{Final numbers for $D^0\to K^+K^-\pi^+\pi^-$ are not public yet}
    \item{Breaking news: $D^0\to\pi^+\pi^-\pi^+\pi^-$ is now available: \href{https://arxiv.org/abs/2408.16279}{arXiv:2408.16279}}
  \end{itemize}
  \vspace{0.3cm}
  \begin{center}
    \textit{\color{blue}During B2OC WG review, the analysis made use of preliminary strong-phase results from the BESIII collaboration. We thank the BESIII management for the privilege of being allowed to show these measurements in internal LHCb meetings. We note that the results for $D^0\to K^+K^-\pi^+\pi^-$ are not yet public, and the results presented on the next slide are not to be shown outside LHCb.}
  \end{center}
\end{frame}

\section{Analysis}
\begin{frame}{Global fit}
  \begin{center}
    {\large Global fit of $K^+K^-\pi^+\pi^-$ remains as in model-dependent publication:}
  \end{center}
  \vspace{-0.5cm}
  \begin{figure}
    \centering
    \includegraphics[width = 0.9\textwidth,trim={0 0 0 0},clip=true]{Plots/d2kkpipi_fiveL_allDP.pdf}
  \end{figure}
  \vspace{-0.5cm}
  \begin{itemize}
    \item{$B^\pm\to[K^+K^-\pi^+\pi^-]_Dh^\pm$ signal yield:}
    \begin{itemize}
      \item{$B^\pm\to DK^\pm$: $3304 \pm 42$}
      \item{$B^\pm\to D\pi^\pm$: $47894 \pm 235$}
    \end{itemize}
  \end{itemize}
\end{frame}

\begin{frame}{Global fit}
  \begin{center}
    {\large Global fit of $\pi^+\pi^-\pi^+\pi^-$ has a good fit quality:}
  \end{center}
  \vspace{-0.5cm}
  \begin{figure}
    \centering
    \includegraphics[width = 0.9\textwidth,trim={0 0 0 0},clip=true]{Plots/d2pipipipi_fiveL_allDP.pdf}
  \end{figure}
  \vspace{-0.5cm}
  \begin{itemize}
    \item{$B^\pm\to[\pi^+\pi^-\pi^+\pi^-]_Dh^\pm$ signal yield:}
    \begin{itemize}
      \item{$B^\pm\to DK^\pm$: $9211 \pm 112$}
      \item{$B^\pm\to D\pi^\pm$: $132654 \pm 398$}
    \end{itemize}
  \end{itemize}
\end{frame}

\begin{frame}{CP fit}
  \begin{center}
    {\large After global fit, perform a ``CP fit'' to study CP violation:}
  \end{center}
  \begin{itemize}
    \setlength\itemsep{1.0em}
    \item{Split candidates by:}
    \begin{enumerate}
      \item{$B^+$ and $B^-$ charges}
      \item{$B^\pm\to DK^\pm$ and $B^\pm\to D\pi^\pm$ decays}
      \item{$D$ phase-space bins}
    \end{enumerate}
    \item{Combinatorial and low-mass backgrounds are floating in each category}
    \item{Parameterise signal yields in terms of $x_\pm^{DK}$, $y_\pm^{DK}$, $x_\xi^{D\pi}$, $y_\xi^{D\pi}$}
    \item{$2N - 1$ floating $F_i$ parameters}
    \item{\underline{$c_i$ and $s_i$ are Gaussian constrained}}
  \end{itemize}
\end{frame}

\begin{frame}{CP fit bin asymmetry}
  \begin{center}
    {\large Example of bin asymmetry in $D\to K^+K^-\pi^+\pi^-$ bin $-3$:}
  \end{center}
  \begin{figure}
    \centering
    \includegraphics[width = 0.9\textwidth,trim={0 10cm 0 0},clip=true]{Plots/d2kkpipi_fiveL_binm3.pdf}
  \end{figure}
\end{frame}

\begin{frame}{CP fit bin asymmetry}
  \begin{center}
    {\large Example of bin asymmetry in $D\to\pi^+\pi^-\pi^+\pi^-$ bin $+5$:}
  \end{center}
  \begin{figure}
    \centering
    \includegraphics[width = 0.9\textwidth,trim={0 10cm 0 0},clip=true]{Plots/d2pipipipi_fiveL_binp5.pdf}
  \end{figure}
\end{frame}

\begin{frame}{Treatment of non-Gaussian uncertainties}
  \begin{center}
    {\large Study of the profile likelihoods show non-Gaussian behaviour induced by $s_i$ uncertainties}
  \end{center}
  \begin{itemize}
    \setlength\itemsep{1.5em}
    \item{This justifies Gaussian constraining $c_i$ and $s_i$}
    \item{Strategy:}
    \begin{enumerate}
      \setlength\itemsep{0.5em}
      \item{Produce a likelihood function from CP fit}
      \item{Interpret CP observables in terms of $\gamma$, etc}
      \item{Must \underline{profile} all nuisance parameters ($F_i$, $c_i$, $s_i$, backgrounds yields, normalisation constants)}
      \item{Provide direct measurements of $\gamma$, $\delta_B$ and $r_B$}
    \end{enumerate}
  \end{itemize}
\end{frame}

\begin{frame}{Interpretation strategy}
  \begin{center}
    {\large From CP fit, we have a (negative log) likelihood function with nuisance parameters $n_k$:}
  \end{center}
  \begin{equation*}
    \mathcal{L}(x_-^{DK}, y_-^{DK}, x_+^{DK}, y_+^{DK}, x_\xi^{D\pi}, y_\xi^{D\pi}, \{n_k\})
  \end{equation*}
  \vspace{0.1cm}
  \begin{center}
    {\large Express in terms of physics parameters:}
  \end{center}
  \begin{equation*}
    \mathcal{L}(\gamma, \delta_B^{DK}, r_B^{DK}, \delta_B^{D\pi}, r_B^{D\pi}, \{n_k\})
  \end{equation*}
  \vspace{0.1cm}
  \begin{center}
    {\normalsize In this step, also add a Gaussian smearing term on CP observables to account for internal LHCb systematics (see backup)}
  \end{center}
\end{frame}

\begin{frame}{Interpretation results}
  \begin{center}
    {\large Results from interpretation of $K^+K^-\pi^+\pi^-$, after correcting for biases in central values (not uncertainties):}
  \end{center}
  \vspace{-0.5cm}
  \begin{columns}
    \begin{column}{0.5\textwidth}
      \begin{center}
        Model independent
      \end{center}
      \begin{align*}
        \gamma =& (119 \pm 14)^\circ \\
        \delta_B^{DK} =& (80 \pm 12)^\circ \\
        r_B^{DK} =& (11.4 \pm 2.3)\times10^{-2} \\
        \delta_B^{D\pi} =& (253 \pm 62)^\circ \\
        r_B^{D\pi} =& (3 \pm 7)\times10^{-3}
      \end{align*}
    \end{column}
    \begin{column}{0.5\textwidth}
      \begin{center}
        Model dependent
      \end{center}
      \begin{align*}
        \gamma =& (116^{+12}_{-14})^\circ \\
        \delta_B^{DK} =& (81^{+14}_{-13})^\circ \\
        r_B^{DK} =& (11.0 \pm 2.0)\times10^{-2} \\
        \delta_B^{D\pi} =& (298^{+62}_{-118})^\circ \\
        r_B^{D\pi} =& (4^{+5}_{-4})\times10^{-3}
      \end{align*}
    \end{column}
  \end{columns}
  \vspace{0.2cm}
  \begin{center}
    Central value of $\gamma$ remains high...\\
    ... it seems that the large tension with the LHCb global result $\gamma = (63.8^{+3.5}_{-3.7})^\circ$ remains
  \end{center}
\end{frame}

\begin{frame}{Interpretation results}
  \begin{center}
    {\large Results from interpretation of $h^+h^-\pi^+\pi^-$, after correcting for biases in central values (not uncertainties):}
  \end{center}
  \vspace{-0.5cm}
  \begin{columns}
    \begin{column}{0.5\textwidth}
      \begin{center}
        $K^+K^-\pi^+\pi^-$
      \end{center}
      \begin{align*}
        \gamma =& (119 \pm 14)^\circ \\
        \delta_B^{DK} =& (80 \pm 12)^\circ \\
        r_B^{DK} =& (11.4 \pm 2.3)\times10^{-2} \\
        \delta_B^{D\pi} =& (253 \pm 62)^\circ \\
        r_B^{D\pi} =& (3 \pm 7)\times10^{-3}
      \end{align*}
    \end{column}
    \begin{column}{0.5\textwidth}
      \begin{center}
        $\pi^+\pi^-\pi^+\pi^-$
      \end{center}
      \begin{align*}
        \gamma =& (45 \pm 9)^\circ \\
        \delta_B^{DK} =& (114 \pm 9)^\circ \\
        r_B^{DK} =& (9.5 \pm 1.9)\times10^{-2} \\
        \delta_B^{D\pi} =& (176 \pm 111)^\circ \\
        r_B^{D\pi} =& (0.8 \pm 1.9)\times10^{-3}
      \end{align*}
    \end{column}
  \end{columns}
  \vspace{0.3cm}
  \begin{center}
    $\pi^+\pi^-\pi^+\pi^-$ is in much better agreement with LHCb global result, but there is a tension with $K^+K^-\pi^+\pi^-$...\\
    \phantom{...but how Gaussian are these uncertainties?}
  \end{center}
\end{frame}

\begin{frame}{Interpretation results}
  \begin{center}
    {\large Results from interpretation of $h^+h^-\pi^+\pi^-$, after correcting for biases in central values (not uncertainties):}
  \end{center}
  \vspace{-0.5cm}
  \begin{columns}
    \begin{column}{0.5\textwidth}
      \begin{center}
        $K^+K^-\pi^+\pi^-$
      \end{center}
      \begin{align*}
        \gamma =& (119 \pm 14)^\circ \\
        \delta_B^{DK} =& (80 \pm 12)^\circ \\
        r_B^{DK} =& (11.4 \pm 2.3)\times10^{-2} \\
        \delta_B^{D\pi} =& (253 \pm 62)^\circ \\
        r_B^{D\pi} =& (3 \pm 7)\times10^{-3}
      \end{align*}
    \end{column}
    \begin{column}{0.5\textwidth}
      \begin{center}
        $\pi^+\pi^-\pi^+\pi^-$
      \end{center}
      \begin{align*}
        \gamma =& (45 \pm 9)^\circ \\
        \delta_B^{DK} =& (114 \pm 9)^\circ \\
        r_B^{DK} =& (9.5 \pm 1.9)\times10^{-2} \\
        \delta_B^{D\pi} =& (176 \pm 111)^\circ \\
        r_B^{D\pi} =& (0.8 \pm 1.9)\times10^{-3}
      \end{align*}
    \end{column}
  \end{columns}
  \vspace{0.3cm}
  \begin{center}
    $\pi^+\pi^-\pi^+\pi^-$ is in much better agreement with LHCb global result, but there is a tension with $K^+K^-\pi^+\pi^-$...\\
    ...but how Gaussian are these uncertainties?
  \end{center}
\end{frame}

\begin{frame}{Likelihood scan of interpretation fit}
  \begin{center}
    In fact, a likelihood scan shows that $D\to K^+K^-\pi^+\pi^-$ and $D\to\pi^+\pi^-\pi^+\pi^-$ $2\sigma$ contours overlap
  \end{center}
  \begin{figure}
    \centering
    \begin{subfigure}{0.50\textwidth}
      \centering
      \includegraphics[width=1.0\textwidth]{Plots/gamma_deltaB_hhpipi_LHCb_Prob_scan.pdf}
      \caption*{$\gamma$ vs $\delta_B^{DK}$}
    \end{subfigure}%
    \begin{subfigure}{0.50\textwidth}
      \centering
      \includegraphics[width=1.0\textwidth]{Plots/rB_deltaB_hhpipi_LHCb_Prob_scan.pdf}
      \caption*{$r_B^{DK}$ vs $\delta_B^{DK}$}
    \end{subfigure}
  \end{figure}
  \vspace{-0.3cm}
  \begin{center}
    When all biases, correlations and non-Gaussian uncertainties are accounted for, the tension with the LHCb average has reduced significantly
  \end{center}
\end{frame}

\begin{frame}{Likelihood scan of interpretation fit}
  \begin{center}
    In fact, a likelihood scan shows that $D\to K^+K^-\pi^+\pi^-$ and $D\to\pi^+\pi^-\pi^+\pi^-$ $2\sigma$ contours overlap
  \end{center}
  \begin{figure}
    \centering
    \begin{subfigure}{0.50\textwidth}
      \centering
      \includegraphics[width=1.0\textwidth]{Plots/gamma_deltaB_hhpipi_LHCb_Prob_scan.pdf}
      \caption*{$\gamma$ vs $\delta_B^{DK}$}
    \end{subfigure}%
    \begin{subfigure}{0.50\textwidth}
      \centering
      \includegraphics[width=1.0\textwidth]{Plots/rB_deltaB_hhpipi_LHCb_Prob_scan.pdf}
      \caption*{$r_B^{DK}$ vs $\delta_B^{DK}$}
    \end{subfigure}
  \end{figure}
  \vspace{-0.3cm}
  \begin{center}
    However, with all the non-Gaussian behaviour, are we sure these contours cover $68\%$ and $95\%$\phantom{y}?
  \end{center}
\end{frame}

\begin{frame}{Plugin/Feldman-Cousins method}
  \begin{center}
    {\Large Feldman-Cousins method, or Plugin, is a ``brute-force'' approach to assigning a confidence interval} \\~\\
    {At each scan point of $\gamma$, perform these fits to data:}
  \end{center}
  \begin{enumerate}
    \setlength\itemsep{0.5em}
    \item{Fit with all parameters floating, and save the log-likelihood $\chi^2$}
    \item{Fit with $\gamma$ fixed to scan point, and save $\chi^2_{\rm fix}$}
    \item{Calculate $\Delta\chi^2_{\rm data} = \chi^2_{\rm fix} - \chi^2$}
  \end{enumerate}
  \vspace{0.5cm}
  \begin{center}
    {We expect $\Delta\chi^2_{\rm data}$ to become large as we move away from best-fit value, but without direct knowledge of underlying PDF, we cannot determine any confidence intervals from this}
  \end{center}
\end{frame}

\begin{frame}{Plugin/Feldman-Cousins method}
  \begin{center}
    {\Large Feldman-Cousins method, or Plugin, is a ``brute-force'' approach to assigning a confidence interval} \\~\\
    {At each scan point of $\gamma$, perform these fits to toy:}
  \end{center}
  \begin{enumerate}
    \setlength\itemsep{0.5em}
    \item{Fix $\gamma$ to scan point and generate $1000$ toys}
    \item{Perform fits to each toy, with $\gamma$ both floating and fixed}
    \item{Calculate $\Delta\chi^2_{\rm toy}$}
  \end{enumerate}
  \vspace{0.42cm}
  \begin{center}
    {At each scan point, the fraction of toys with $\Delta\chi^2_{\rm toy} > \Delta\chi^2_{\rm data}$ is equal to $1 - \rm{CL}$, and the exact $68\%$ confidence interval can then be obtained using an interpolation between points}
  \end{center}
\end{frame}

\begin{frame}{Plugin/Feldman-Cousins method}
  \begin{center}
    LHCb average within $2\sigma$ of $D\to K^+K^-\pi^+\pi^-$ Plugin result \\
    Combined fit shows good agreement between Plugin and Prob scans
  \end{center}
  \begin{figure}
    \centering
    \includegraphics[width=0.6\textwidth]{Plots/gamma_plugin_scan.pdf}
  \end{figure}
  \vspace{-0.3cm}
  \begin{center}
    Combined fit result: $\gamma = (54.0_{-9.5}^{+10.2})^\circ$ \\
    Third most precise single measurement of $\gamma$ in $B^\pm$ decays
  \end{center}
\end{frame}

\section{Main changes during WG review}
\begin{frame}{Main changes during WG review}
  \begin{center}
    {\Large Analysis progression since last presentation:}
  \end{center}
  \begin{enumerate}
    \setlength\itemsep{0.8em}
    \item{Selection between Oxford and Bristol groups have been merged}
    \item{Many selection cuts are now applied before BDT, instead of after}
    \item{All internal LHCb internal systematic uncertainties evaluated}
    \item{Combination of phase-space binned and integrated results}
  \end{enumerate}
  \vspace{0.3cm}
  \begin{center}
    {\Large Important points that were discussed during review:}
  \end{center}
  \begin{itemize}
    \setlength\itemsep{0.8em}
    \item{Correlation between binned and integrated results}
    \item{Efficiency corrections to $c_i$ and $s_i$}
  \end{itemize}
\end{frame}

\begin{frame}{Changing order of cuts}
  \begin{center}
    {\Large Selection ``workflow'' before review:}
  \end{center}
  \vspace{0.4cm}
  \begin{columns}
    \begin{column}{0.33\textwidth}
      1. Initial cuts
      \begin{itemize}
        \item{Triggers}
        \item{RICH information}
        \item{Mass range}
        \item{Etc}
      \end{itemize}
    \end{column}
    \begin{column}{0.33\textwidth}
      2. BDT
      \begin{itemize}
        \item{Combinatorial background}
        \item{Optimised for $\gamma$ sensitivity}
      \end{itemize}
    \end{column}
    \begin{column}{0.33\textwidth}
      3. Final cuts
      \begin{itemize}
        \item{PID}
        \item{Flight significance}
        \item{Opening angle}
        \item{$K_S^0$ veto}
      \end{itemize}
    \end{column}
  \end{columns}
\end{frame}

\begin{frame}{Changing order of cuts}
  \begin{center}
    {\Large Selection ``workflow'' after review:}
  \end{center}
  \vspace{0.4cm}
  \begin{columns}
    \begin{column}{0.33\textwidth}
      1. Initial cuts
      \begin{itemize}
        \item{Triggers}
        \item{RICH information}
        \item{Mass range}
        \item{Flight significance}
        \item{Opening angle}
        \item{$K_S^0$ veto}
      \end{itemize}
    \end{column}
    \begin{column}{0.33\textwidth}
      2. BDT
      \begin{itemize}
        \item{Combinatorial background}
        \item{Optimised for $\gamma$ sensitivity}
      \end{itemize}
    \end{column}
    \begin{column}{0.33\textwidth}
      3. Final cuts
      \begin{itemize}
        \item{PID}
      \end{itemize}
    \end{column}
  \end{columns}
  \vspace{0.2cm}
  \begin{center}
    {\Large Minimal change in final results}
  \end{center}
\end{frame}

\begin{frame}{Summary of LHCb internal systematic uncertainties}
  \scriptsize
  \vspace{0.02cm}
  \begin{center}
    \begin{tabular}{lcccccc}
      \hline
      Source & $x_-^{DK}$ & $y_-^{DK}$ & $x_+^{DK}$ & $y_+^{DK}$ & $x_\xi^{D\pi}$ & $y_\xi^{D\pi}$ \\
      \hline
      Statistical                                                & $2.87$ & $3.40$ & $2.51$ & $3.05$ & $4.24$ & $5.17$ \\
      \hline
      Mass shape                                                 & $0.02$ & $0.02$ & $0.03$ & $0.06$ & $0.02$ & $0.04$ \\
      Bin-dependent mass shape                                   & $0.11$ & $0.05$ & $0.10$ & $0.19$ & $0.68$ & $0.16$ \\
      PID efficiency                                             & $0.02$ & $0.02$ & $0.03$ & $0.06$ & $0.02$ & $0.04$ \\
      Low-mass background model                                  & $0.02$ & $0.02$ & $0.03$ & $0.04$ & $0.02$ & $0.02$ \\
      Charmless background                                       & $0.14$ & $0.15$ & $0.12$ & $0.14$ & $0.01$ & $0.02$ \\
      $C\!P$ violation in low-mass background                    & $0.01$ & $0.10$ & $0.08$ & $0.12$ & $0.07$ & $0.26$ \\
      Semi-leptonic $b$-hadron decays                            & $0.05$ & $0.27$ & $0.06$ & $0.01$ & $0.07$ & $0.19$ \\
      Semi-leptonic charm decays                                 & $0.02$ & $0.07$ & $0.03$ & $0.15$ & $0.06$ & $0.24$ \\
      $D\to K^-\pi^+\pi^-\pi^+$ background                       & $0.11$ & $0.05$ & $0.07$ & $0.04$ & $0.09$ & $0.05$ \\
      $\Lambda_b\to pD\pi^-$ background                          & $0.01$ & $0.25$ & $0.14$ & $0.04$ & $0.06$ & $0.34$ \\
      $D\to K^-\pi^+\pi^-\pi^+\pi^0$ background                  & $0.30$ & $0.05$ & $0.19$ & $0.07$ & $0.05$ & $0.01$ \\
      Fit bias                                                   & $0.06$ & $0.05$ & $0.13$ & $0.02$ & $0.06$ & $0.13$ \\
      \hline
      Total LHCb systematic                                      & $0.37$ & $0.43$ & $0.34$ & $0.32$ & $0.70$ & $0.57$ \\
      \hline
    \end{tabular}
  \end{center}
  \begin{center}
    {\normalsize Numbers for $\pi^+\pi^-\pi^+\pi^-$ are very similar}
  \end{center}
\end{frame}

\begin{frame}{Phase-space integrated CP observables}
  \begin{center}
    {\large Phase-space integrated study of $\gamma$: \\
    Charged asymmetries measured for $D\to K^+K^-\pi^+\pi^-$ and $D\to\pi^+\pi^-\pi^+\pi^-$ in Eur. Phys. J. C \textbf{83} 547 (2023)}
  \end{center}
  \vspace{-0.4cm}
  \begin{figure}
    \centering
    \includegraphics[width = 0.9\textwidth,trim={0 7cm 0 0},clip=true]{Plots/d2kkpipi_fiveL_allDP_GLW.pdf}
    \caption*{$D\to K^+K^-\pi^+\pi^-$}
  \end{figure}
  \vspace{-0.5cm}
  \begin{itemize}
    \item{$B^\pm\to[h^+h^-\pi^+\pi^-]_Dh^\pm$ asymmetries:}
    \begin{itemize}
      \item[-]{$D\to K^+K^-\pi^+\pi^-$: $\mathcal{A} = 0.095 \pm 0.023 \pm 0.002$}
      \item[-]{$D\to\pi^+\pi^-\pi^+\pi^-$: $\mathcal{A} = 0.061 \pm 0.013 \pm 0.002$}
    \end{itemize}
  \end{itemize}
\end{frame}

\begin{frame}{Phase-space integrated CP observables}
  \begin{center}
    {\large Phase-space integrated study of $\gamma$: \\
    Charged asymmetries measured for $D\to K^+K^-\pi^+\pi^-$ and $D\to\pi^+\pi^-\pi^+\pi^-$ in Eur. Phys. J. C \textbf{83} 547 (2023)}
  \end{center}
  \vspace{-0.4cm}
  \begin{figure}
    \centering
    \includegraphics[width = 0.9\textwidth,trim={0 7cm 0 0},clip=true]{Plots/d2pipipipi_fiveL_allDP_GLW.pdf}
    \caption*{$D\to\pi^+\pi^-\pi^+\pi^-$}
  \end{figure}
  \vspace{-0.5cm}
  \begin{itemize}
    \item{$B^\pm\to[h^+h^-\pi^+\pi^-]_Dh^\pm$ asymmetries:}
    \begin{itemize}
      \item[-]{$D\to K^+K^-\pi^+\pi^-$: $\mathcal{A} = 0.095 \pm 0.023 \pm 0.002$}
      \item[-]{$D\to\pi^+\pi^-\pi^+\pi^-$: $\mathcal{A} = 0.061 \pm 0.013 \pm 0.002$}
    \end{itemize}
  \end{itemize}
\end{frame}

\begin{frame}{Combining phase-space binned and integrated results}
  \begin{center}
    We can add phase-space integrated observables as a constraint:
  \end{center}
  \begin{figure}
    \centering
    \begin{subfigure}{0.50\textwidth}
      \centering
      \includegraphics[width=1.0\textwidth]{Plots/GLW_constraints_comparison_gamma_deltaB.pdf}
      \caption*{$\gamma$ vs $\delta_B^{DK}$}
    \end{subfigure}%
    \begin{subfigure}{0.50\textwidth}
      \centering
      \includegraphics[width=1.0\textwidth]{Plots/GLW_constraints_comparison_rB_deltaB.pdf}
      \caption*{$r_B^{DK}$ vs $\delta_B^{DK}$}
    \end{subfigure}
  \end{figure}
  \vspace{-0.3cm}
  \begin{center}
    The global asymmetries contain useful information!
  \end{center}
\end{frame}

\begin{frame}{Combining phase-space binned and integrated results}
  \begin{center}
    Run Plugin with phase-space integrated constraints:
  \end{center}
  \begin{figure}
    \centering
    \includegraphics[width=0.6\textwidth]{Plots/gamma_plugin_scan_GLW.pdf}
  \end{figure}
  \vspace{-0.3cm}
  \begin{center}
    Final measurement: $\gamma = (51.2_{-6.5}^{+8.9})^\circ$ \\
  \end{center}
\end{frame}

\section{Conclusion and future prospects}
\begin{frame}{Conclusion}
  \vspace{-0.1cm}
  \begin{enumerate}
    \setlength\itemsep{2.0em}
    \item{Binned model-independent measurement of $\gamma$ with $B^\pm\to[h^+h^-\pi^+\pi^-]_Dh^\pm$ has been performed: $\gamma = (54.0_{-9.5}^{+10.2})^\circ$}
    \begin{itemize}
      \item{External strong-phase inputs from BESIII}
    \end{itemize}
    \item{Combination with integrated results: $\gamma = (51.2_{-6.5}^{+8.9})^\circ$}
    \item{$3\sigma$ tension in $D\to K^+K^-\pi^+\pi^-$ has reduced}
  \end{enumerate}
\end{frame}

\begin{frame}{Future prospects}
  \vspace{0.6cm}
  \begin{itemize}
    \setlength\itemsep{2.0em}
    \item{We see no showstoppers for this analysis, which will provide important constraints to GammaCombo in the near future}
    \item{BESIII results for $K^+K^-\pi^+\pi^-$ will be available imminently}
    \item{We would like this analysis to move to RC}
    \begin{itemize}
      \item{You can find the TWiki \href{https://twiki.cern.ch/twiki/bin/viewauth/LHCbPhysics/BPGGSZB2DhD2hhpipiModelIndependent}{here}}
    \end{itemize}
  \end{itemize}
  \vspace{0.5cm}
  \begin{center}
    {\huge Thanks for your attention!}
  \end{center}
\end{frame}

\section{Strong-phase inputs from BESIII}
\begin{frame}{Backup: BESIII preliminary $D^0\to K^+K^-\pi^+\pi^-$ strong-phase results}
  \begin{center}
    First binned strong-phase analysis of $D^0\to K^+K^-\pi^+\pi^-$, which uses the $2\times4$ binning scheme with $16$~fb$^{-1}$ $\psi(3770)$ data
  \end{center}
  \vspace{-0.3cm}
  \begin{columns}
    \begin{column}{0.5\textwidth}
      \vspace{-0.5cm}
      \begin{align*}
        c_1 =& -0.28 \pm 0.09 \pm 0.01 \\
        s_1 =& -0.68 \pm 0.24 \pm 0.04 \\
        c_2 =& +0.83 \pm 0.04 \pm 0.01 \\
        s_2 =& -0.18 \pm 0.19 \pm 0.03 \\
        c_3 =& +0.83 \pm 0.03 \pm 0.01 \\
        s_3 =& +0.27 \pm 0.17 \pm 0.03 \\
        c_4 =& -0.28 \pm 0.10 \pm 0.01 \\
        s_4 =& +0.54 \pm 0.28 \pm 0.04
      \end{align*}
    \end{column}
    \begin{column}{0.5\textwidth}
      \begin{figure}
        \centering
        \includegraphics[width=0.9\textwidth]{Plots/cisi_FitResults_Model.pdf}
      \end{figure}
    \end{column}
  \end{columns}
  \begin{center}
    Measured values (black) are consistent and close to LHCb model predictions (blue), so central values are not expected to change much
  \end{center}
\end{frame}

\begin{frame}{Backup: BESIII preliminary $D^0\to\pi^+\pi^-\pi^+\pi^-$ strong-phase results}
  \begin{center}
    Small differences between model prediction and measurement, but data points are generally close to the unit circle
  \end{center}
  \vspace{-0.3cm}
  \begin{columns}
    \begin{column}{0.50\textwidth}
      \vspace{-0.5cm}
      \begin{align*}
        c_1 =& +0.12 \pm 0.09 \pm 0.02 \\
        s_1 =& -0.42 \pm 0.21 \pm 0.04 \\
        c_2 =& +0.74 \pm 0.04 \pm 0.02 \\
        s_2 =& -0.39 \pm 0.16 \pm 0.06 \\
        s_3 =& -0.25 \pm 0.12 \pm 0.03 \\
        c_3 =& +0.81 \pm 0.03 \pm 0.01 \\
        c_4 =& +0.42 \pm 0.06 \pm 0.02 \\
        s_4 =& +0.86 \pm 0.19 \pm 0.07 \\
        c_5 =& -0.27 \pm 0.09 \pm 0.03 \\
        s_5 =& -0.22 \pm 0.25 \pm 0.08
      \end{align*}
    \end{column}
    \begin{column}{0.50\textwidth}
      \begin{figure}
        \centering
        \includegraphics[width=1.0\textwidth]{Plots/CiSiOptim.pdf}
      \end{figure}
    \end{column}
  \end{columns}
  \begin{center}
    The HyperPlot software is used (binary lookup tree in 5D phase space)
  \end{center}
\end{frame}

\begin{frame}{Backup: BESIII preliminary $D^0\to\pi^+\pi^-\pi^+\pi^-$ strong-phase results}
  \vspace{0.3cm}
  \begin{itemize}
    \setlength\itemsep{1.0em}
    \item{Binned strong-phase analysis of $D^0\to\pi^+\pi^-\pi^+\pi^-$ uses the $2\times5$ ``optimal'' binning scheme with $3$~fb$^{-1}$ $\psi(3770)$}
    \item{Earlier CLEO-c analysis with $0.8$~fb$^{-1}$ \href{https://link.springer.com/article/10.1007/JHEP01(2018)144}{JHEP \textbf{01} (2018) 144}}
    \item{New BESIII analysis uses a new binning scheme optimised with a BESIII amplitude model \href{https://arxiv.org/abs/2312.02524}{arXiv:2312.02524}}
    \begin{itemize}
      \item{Amplitude model constructed from a larger data set}
      \item{In principle more sensitive}
    \end{itemize}
    \item{Two binning schemes are available:}
    \begin{itemize}
      \item{We use the more sensitive ``optimal'' binning with $Q = 0.85$}
      \item{The other ``equal $\delta$'' binning has $Q = 0.80$}
    \end{itemize}
  \end{itemize}
\end{frame}

\begin{frame}{Backup: Global fit}
  \begin{center}
    {\large How do we include the $\pi^+\pi^-\pi^+\pi^-$ mode?}
  \end{center}
  \begin{itemize}
    \setlength\itemsep{1.5em}
    \item{We \underline{have already studied} $B^\pm\to[\pi^+\pi^-\pi^+\pi^-]_Dh^\pm$ for phase-space integrated measurement}
    \begin{enumerate}
      \setlength\itemsep{0.5em}
      \item{Different $D$ daughter PID cuts in stripping}
      \item{No $D\to K\pi\pi\pi\pi^0$ background}
      \item{Charmless background recalculated using the sideband}
      \item{Use same BDT}
      \item{No additional peaking backgrounds}
    \end{enumerate}
    \item{Sort candidates into phase-space bins using BESIII binning scheme}
    \item{Can fit separately or simultaneously with $K^+K^-\pi^+\pi^-$}
  \end{itemize}
\end{frame}

\begin{frame}{Backup: Strong-phase parameters in CP fit}
  \begin{center}
    {\large Why are $c_i$ and $s_i$ Gaussian constrained?}
  \end{center}
  \begin{itemize}
    \setlength\itemsep{1.0em}
    \item{Previous BPGGSZ analyses have kept $c_i$ and $s_i$ fixed}
    \begin{enumerate}
      \item{$c_i$ and $s_i$ uncertainties are added as a systematic through smearing}
      \item{Convenient for calculating correlations between different analyses}
      \item{Appropriate when $c_i$ and $s_i$ uncertainties are \underline{small}}
    \end{enumerate}
    \item{In four-body analyses, uncertainties on $\gamma$ from $c_i$ and $s_i$ are almost the same size as the statistical uncertainty}
    \item{Large $s_i$ uncertainties introduces non-Gaussian uncertainties on $y_\pm$}
    \item{$\gamma$/$\delta_B$ move significantly when fixing $s_i$ instead of constraining them}
    \item{These effects are largest for $K^+K^-\pi^+\pi^-$, but are also seen in $\pi^+\pi^-\pi^+\pi^-$ and in the combined fit}
  \end{itemize}
\end{frame}

\begin{frame}{Backup: CP fit categories}
  \begin{center}
    {\large Summary of free parameters in the CP fit:}
  \end{center}
  \vspace{-0.5cm}
  \begin{columns}
    \begin{column}{0.5\textwidth}
      \begin{center}
        $K^+K^-\pi^+\pi^-$ \\
        $2\times2\times2\times4 = 32$ categories
      \end{center}
      \begin{itemize}
        \setlength\itemsep{0.0em}
        \item{6 CP observables}
        \item{7 $F_i$ parameters}
        \item{8 $c_i$ and $s_i$ parameters}
        \item{32 combinatorial yields}
        \item{32 low mass yields}
        \item{4 global normalisations}
        \item{Total: 89 parameters}
      \end{itemize}
    \end{column}
    \begin{column}{0.5\textwidth}
      \begin{center}
        $\pi^+\pi^-\pi^+\pi^-$ \\
        $2\times2\times2\times5 = 40$ categories
      \end{center}
      \begin{itemize}
        \setlength\itemsep{0.0em}
        \item{6 CP observables}
        \item{9 $F_i$ parameters}
        \item{10 $c_i$ and $s_i$ parameters}
        \item{40 combinatorial yields}
        \item{40 low mass yields}
        \item{4 global normalisations}
        \item{Total: 109 parameters}
      \end{itemize}
    \end{column}
  \end{columns}
  \vspace{0.3cm}
  \begin{center}
    In a combined fit where CP observables are shared, there are $89 + 109 - 6 = 192$ parameters
  \end{center}
\end{frame}

\begin{frame}{Backup: Bin asymmetries}
  \begin{center}
    $B^\pm\to[K^+K^-\pi^+\pi^-]_Dh^\pm$ bin asymmetries
  \end{center}
  \begin{figure}
    \centering
    \begin{subfigure}{0.5\textwidth}
      \centering
      \includegraphics[width=1.0\textwidth]{Plots/BinAsymmetries_dk_KKpipi.pdf}
      \caption*{$B^\pm\to DK^\pm$}
    \end{subfigure}%
    \begin{subfigure}{0.5\textwidth}
      \centering
      \includegraphics[width=1.0\textwidth]{Plots/BinAsymmetries_dpi_KKpipi.pdf}
      \caption*{$B^\pm\to D\pi^\pm$}
    \end{subfigure}
  \end{figure}
\end{frame}

\begin{frame}{Backup: Bin asymmetries}
  \begin{center}
    $B^\pm\to[\pi^+\pi^-\pi^+\pi^-]_Dh^\pm$ bin asymmetries
  \end{center}
  \begin{figure}
    \centering
    \begin{subfigure}{0.5\textwidth}
      \centering
      \includegraphics[width=1.0\textwidth]{Plots/BinAsymmetries_dk_pipipipi.pdf}
      \caption*{$B^\pm\to DK^\pm$}
    \end{subfigure}%
    \begin{subfigure}{0.5\textwidth}
      \centering
      \includegraphics[width=1.0\textwidth]{Plots/BinAsymmetries_dpi_pipipipi.pdf}
      \caption*{$B^\pm\to D\pi^\pm$}
    \end{subfigure}
  \end{figure}
\end{frame}

\begin{frame}{Backup: CP fit toy studies}
  \begin{center}
    In toy studies biases in $D\pi$ observables are consistent with model-dependent analysis
  \end{center}
  \begin{figure}
    \centering
    \begin{subfigure}{0.5\textwidth}
      \centering
      \includegraphics[width=1.0\textwidth]{Plots/A_Re_xi_dpi_pull.pdf}
      \vspace{-0.3cm}
      \caption*{$x_\xi^{D\pi}$}
    \end{subfigure}%
    \begin{subfigure}{0.5\textwidth}
      \centering
      \includegraphics[width=1.0\textwidth]{Plots/A_Im_xi_dpi_pull.pdf}
      \vspace{-0.3cm}
      \caption*{$y_\xi^{D\pi}$}
    \end{subfigure}
    \caption*{$D^0\to K^+K^-\pi^+\pi^-$}
  \end{figure}
\end{frame}

\begin{frame}{Backup: CP fit toy studies}
  \begin{center}
    Minor biases in $x_\pm^{DK}$ are seen but can be corrected for...
  \end{center}
  \begin{figure}
    \centering
    \begin{subfigure}{0.5\textwidth}
      \centering
      \includegraphics[width=1.0\textwidth]{Plots/A_xm_dk_pull.pdf}
      \vspace{-0.3cm}
      \caption*{$x_-^{DK}$}
    \end{subfigure}%
    \begin{subfigure}{0.5\textwidth}
      \centering
      \includegraphics[width=1.0\textwidth]{Plots/A_xp_dk_pull.pdf}
      \vspace{-0.3cm}
      \caption*{$x_+^{DK}$}
    \end{subfigure}
    \caption*{$D^0\to K^+K^-\pi^+\pi^-$}
  \end{figure}
\end{frame}

\begin{frame}{Backup: CP fit toy studies}
  \begin{center}
    ...but $y_\pm^{DK}$ pulls are now slightly asymmetric!
  \end{center}
  \begin{figure}
    \centering
    \begin{subfigure}{0.5\textwidth}
      \centering
      \includegraphics[width=1.0\textwidth]{Plots/A_ym_dk_pull.pdf}
      \vspace{-0.3cm}
      \caption*{$y_-^{DK}$}
    \end{subfigure}%
    \begin{subfigure}{0.5\textwidth}
      \centering
      \includegraphics[width=1.0\textwidth]{Plots/A_yp_dk_pull.pdf}
      \vspace{-0.3cm}
      \caption*{$y_+^{DK}$}
    \end{subfigure}
    \caption*{$D^0\to K^+K^-\pi^+\pi^-$}
  \end{figure}
\end{frame}

\begin{frame}{Backup: Likelihood scan of CP observables}
  \begin{center}
    $x_\pm^{DK}$ agree well between likelihood scan and Hesse approximation
  \end{center}
  \begin{figure}
    \centering
    \begin{subfigure}{0.5\textwidth}
      \centering
      \includegraphics[width=1.0\textwidth]{Plots/A_xm_dk_likelihood_scan_KKpipi.pdf}
      \vspace{-0.3cm}
      \caption*{$x_-^{DK}$}
    \end{subfigure}%
    \begin{subfigure}{0.5\textwidth}
      \centering
      \includegraphics[width=1.0\textwidth]{Plots/A_xp_dk_likelihood_scan_KKpipi.pdf}
      \vspace{-0.3cm}
      \caption*{$x_+^{DK}$}
    \end{subfigure}
    \caption*{$D^0\to K^+K^-\pi^+\pi^-$}
  \end{figure}
\end{frame}

\begin{frame}{Backup: Likelihood scan of CP observables}
  \begin{center}
    $y_\pm^{DK}$ diverges from Hesse approximation outside $1\sigma$
  \end{center}
  \begin{figure}
    \centering
    \begin{subfigure}{0.5\textwidth}
      \centering
      \includegraphics[width=1.0\textwidth]{Plots/A_ym_dk_likelihood_scan_KKpipi.pdf}
      \vspace{-0.3cm}
      \caption*{$y_-^{DK}$}
    \end{subfigure}%
    \begin{subfigure}{0.5\textwidth}
      \centering
      \includegraphics[width=1.0\textwidth]{Plots/A_yp_dk_likelihood_scan_KKpipi.pdf}
      \vspace{-0.3cm}
      \caption*{$y_+^{DK}$}
    \end{subfigure}
    \caption*{$D^0\to K^+K^-\pi^+\pi^-$}
  \end{figure}
\end{frame}

\begin{frame}{Backup: Likelihood scan of CP observables}
  \begin{center}
    $x_\pm^{DK}$ agree well between likelihood scan and Hesse approximation
  \end{center}
  \begin{figure}
    \centering
    \begin{subfigure}{0.5\textwidth}
      \centering
      \includegraphics[width=1.0\textwidth]{Plots/A_xm_dk_likelihood_scan_pipipipi.pdf}
      \vspace{-0.3cm}
      \caption*{$x_-^{DK}$}
    \end{subfigure}%
    \begin{subfigure}{0.5\textwidth}
      \centering
      \includegraphics[width=1.0\textwidth]{Plots/A_xp_dk_likelihood_scan_pipipipi.pdf}
      \vspace{-0.3cm}
      \caption*{$x_+^{DK}$}
    \end{subfigure}
    \caption*{$D^0\to\pi^+\pi^-\pi^+\pi^-$}
  \end{figure}
\end{frame}

\begin{frame}{Backup: Likelihood scan of CP observables}
  \begin{center}
    $y_\pm^{DK}$ diverges from Hesse approximation outside $1\sigma$
  \end{center}
  \begin{figure}
    \centering
    \begin{subfigure}{0.5\textwidth}
      \centering
      \includegraphics[width=1.0\textwidth]{Plots/A_ym_dk_likelihood_scan_pipipipi.pdf}
      \vspace{-0.3cm}
      \caption*{$y_-^{DK}$}
    \end{subfigure}%
    \begin{subfigure}{0.5\textwidth}
      \centering
      \includegraphics[width=1.0\textwidth]{Plots/A_yp_dk_likelihood_scan_pipipipi.pdf}
      \vspace{-0.3cm}
      \caption*{$y_+^{DK}$}
    \end{subfigure}
    \caption*{$D^0\to\pi^+\pi^-\pi^+\pi^-$}
  \end{figure}
\end{frame}

\begin{frame}{Backup: Interpretation toys}
  \begin{center}
    We can perform toy studies on the interpretation fit, but we do \underline{not} expect these to behave very Gaussian...
  \end{center}
  \begin{figure}
    \centering
    \begin{subfigure}{0.5\textwidth}
      \centering
      \includegraphics[width=1.0\textwidth]{Plots/gamma_pull_toys_KKpipi.pdf}
      \vspace{-0.3cm}
      \caption*{$K^+K^-\pi^+\pi^-$}
    \end{subfigure}%
    \begin{subfigure}{0.5\textwidth}
      \centering
      \includegraphics[width=1.0\textwidth]{Plots/gamma_pull_toys_pipipipi.pdf}
      \vspace{-0.3cm}
      \caption*{$\pi^+\pi^-\pi^+\pi^-$}
    \end{subfigure}
    \vspace{-0.5cm}
    \caption*{$\gamma$ pull distributions}
  \end{figure}
  \vspace{-0.3cm}
  \begin{center}
    Indeed, small but significant biases are observed!\\
    Use pull distributions to correct central values of physics parameters
  \end{center}
\end{frame}

\begin{frame}{Backup: Interpretation toys}
  \begin{center}
    We can perform toy studies on the interpretation fit, but we do \underline{not} expect these to behave very Gaussian...
  \end{center}
  \begin{figure}
    \centering
    \begin{subfigure}{0.5\textwidth}
      \centering
      \includegraphics[width=1.0\textwidth]{Plots/gamma_pull_toys_KKpipi.pdf}
      \vspace{-0.3cm}
      \caption*{$K^+K^-\pi^+\pi^-$}
    \end{subfigure}%
    \begin{subfigure}{0.5\textwidth}
      \centering
      \includegraphics[width=1.0\textwidth]{Plots/gamma_pull_toys_pipipipi.pdf}
      \vspace{-0.3cm}
      \caption*{$\pi^+\pi^-\pi^+\pi^-$}
    \end{subfigure}
    \vspace{-0.5cm}
    \caption*{$\gamma$ pull distributions}
  \end{figure}
  \vspace{-0.3cm}
  \begin{center}
    The absolute bias corrections are:\\
    $K^+K^-\pi^+\pi^-$: $+5.6^\circ$, $\pi^+\pi^-\pi^+\pi-$: $-3.0^\circ$, combined: $-3.0^\circ$
  \end{center}
\end{frame}

\begin{frame}{Backup: Charm mixing studies with multi-body decays}
  \begin{center}
    {\large Sensitivity to $c_i$: Similar between BESIII and charm mixing at LHCb}
  \end{center}
  \begin{figure}[htb]
    \centering
    \includegraphics[width=0.7\textwidth]{Plots/ci_sensitivity.pdf}
  \end{figure}
  \vspace{-0.4cm}
  \begin{itemize}
    \item{BESIII yields equivalent to $\SI{8}{\per\femto\barn}$ of $\psi(3770)$}
    \item{4 million $D\to K^+K^-\pi^+\pi^-$ candidates in mixing analysis}
    \end{itemize}
\end{frame}

\begin{frame}{Backup: Charm mixing studies with multi-body decays}
  \begin{center}
    {\large Sensitivity to $s_i$: Significant improvements expected!}
  \end{center}
  \begin{figure}[htb]
    \centering
    \includegraphics[width=0.7\textwidth]{Plots/si_sensitivity.pdf}
  \end{figure}
  \vspace{-0.4cm}
  \begin{itemize}
    \item{BESIII yields equivalent to $\SI{8}{\per\femto\barn}$ of $\psi(3770)$}
    \item{4 million $D\to K^+K^-\pi^+\pi^-$ candidates in mixing analysis}
  \end{itemize}
\end{frame}

\end{document}
